% ****** Start of file apssamp.tex ******
%
%   This file is part of the APS files in the REVTeX 4.1 distribution.
%   Version 4.1r of REVTeX, August 2010
%
%   Copyright (c) 2009, 2010 The American Physical Society.
%
%   See the REVTeX 4 README file for restrictions and more information.
%
% TeX'ing this file requires that you have AMS-LaTeX 2.0 installed
% as well as the rest of the prerequisites for REVTeX 4.1
%
% See the REVTeX 4 README file
% It also requires running BibTeX. The commands are as follows:
%
%  1)  latex apssamp.tex
%  2)  bibtex apssamp
%  3)  latex apssamp.tex
%  4)  latex apssamp.tex
%
\documentclass[%
reprint,
superscriptaddress,
%groupedaddress,
%unsortedaddress,
%runinaddress,
%frontmatterverbose,
%preprint,
showpacs,preprintnumbers,
%nofootinbib,
%nobibnotes,
%bibnotes,
 amsmath,amssymb,
 aps,
%pra,
%prb,
prd,
%prl,
%rmp,
%prstab,
%prstper,
%floatfix,
]{revtex4-1}

\usepackage{graphicx}% Include figure files
\usepackage{dcolumn}% Align table columns on decimal point
\usepackage{bm}% bold math
\usepackage{bbold}
\usepackage{amssymb,amsmath}
\usepackage{hyperref}% add hypertext capabilities
%\usepackage[mathlines]{lineno}% Enable numbering of text and display math
%\linenumbers\relax % Commence numbering lines

%\usepackage[showframe,%Uncomment any one of the following lines to test
%%scale=0.7, marginratio={1:1, 2:3}, ignoreall,% default settings
%%text={7in,10in},centering,
%%margin=1.5in,
%%total={6.5in,8.75in}, top=1.2in, left=0.9in, includefoot,
%%height=10in,a5paper,hmargin={3cm,0.8in},
%]{geometry}

\usepackage{color}
\usepackage{amsfonts}
\usepackage{subfigure}
\usepackage{array}

\newcommand{\Tr}{\ensuremath{\operatorname{Tr}}}
\newcommand{\tr}{\ensuremath{\operatorname{tr}}}
\newcommand{\Omegaqq}{\ensuremath{\Omega_{\bar{q}q}}}
\newcommand{\vev}[1]{\ensuremath{\left\langle #1 \right\rangle}}
\newcommand{\einh}[1]{\ensuremath{\,\text{#1}}}
\newcolumntype{L}{>{\centering\arraybackslash}m{3cm}}

\newcommand{\overbar}[1]{\mkern 1.5mu\overline{\mkern-1.5mu#1\mkern-1.5mu}\mkern 1.5mu}

\definecolor{bjcol}{rgb}{1,.44,0.13}

% color def's

\definecolor{blue}{rgb}{0,0,1}
\newcommand{\colb}[1]{{\color{blue} #1}}
\definecolor{green}{rgb}{0,1,0}
\newcommand{\colg}[1]{{\color{green} #1}}
\definecolor{red}{rgb}{1,0,0}
\newcommand{\colr}[1]{{\color{red} #1}}
\newcommand{\colJ}[1]{{\color{cyan} #1}}
\definecolor{gray}{rgb}{.5,.5,.5}
\newcommand{\drop}[1]{{\sout{ {\color{gray} #1}}}}
\definecolor{darkgreen}{rgb}{.0,.5,.0}
\newcommand{\colL}[1]{{\color{darkgreen} #1}}

\def\Fig#1{Fig.~\ref{#1}} \def\Tab#1{Tab.~\ref{#1}}
\def\Figs#1{Figs.~\ref{#1}} \def\Tab#1{Tab.~\ref{#1}}
\def\Eqs#1{Eqs.~(\ref{#1})}
\def\Eq#1{Eq.~(\ref{#1})}
\def\eq#1{(\ref{#1})}
\def\eqref#1{(\ref{#1})}
\def\fig#1{Fig.~\ref{#1}}
\def\tab#1{Tab.~\ref{#1}}
\def\eqs#1{(\ref{#1})}
\def\Eqs#1{(\ref{#1})}
\def\sec#1{Sec.~\ref{#1}}
\def\app#1{Appendix~\ref{#1}}
\newcommand{\Phibar}{\ensuremath{\bar{\Phi}}}
\newcommand{\LPQM}{\ensuremath{\mathcal{L}_{\textrm{PQM}}}\xspace}

\def\dbar{{\mathchar'26\mkern-12mu d}}
\def\lA0{{\langle A_0 \rangle}}
\def\bA0{{\bar{A}_0}}
\def\lLA{{\langle L[A_0] \rangle}}
\def\lL{{\langle L \rangle}}
\def\lLc{{\langle L^\dagger \rangle}}
\def\lLAc{{\langle L^\dagger[A_0] \rangle}}

\def\dr{{D\!\llap{/}}\,}
\def\Dr{{D\!\llap{/}}\,}
\def\ipv{\vec{p}\llap{/}}
\def\pslash{p\llap{/}}

\def\0#1#2{\frac{#1}{#2}}

\newcommand{\bsig}{\ensuremath{\bar{\sigma}}}
\newcommand{\lsm}{L\ensuremath{\sigma}M\xspace}
\newcommand{\pT}{\ensuremath{T_0}}
\newcommand{\Tl}{\ensuremath{T_\chi}}
\newcommand{\Ts}{\ensuremath{T_\chi^s}}
\newcommand{\Tchi}{\ensuremath{T_\chi}}
\newcommand{\Td}{\ensuremath{T_d}}
\newcommand{\Tc}{\ensuremath{T_c}}
\newcommand{\muc}{\ensuremath{\mu_c}}
\newcommand{\coloronl}{(color online)\xspace}

\newcommand{\mrm}[1]{\mathrm{#1}}
\def\qbar{\bar{q}}
\newcommand{\sx}{\sigma_{x}}
\newcommand{\sy}{\sigma_{y}}

%%%%%%%%%%%%%% for corrections %%%%%%%%%%%
\newcommand{\colrw}[1]{\textcolor{blue}{#1}}
\newcommand{\colcyr}[1]{\textcolor{green}{#1}}
\newcommand{\colwjf}[1]{\textcolor{red}{#1}}

%
%%%%%%%%%%%%%%%%%%%%%%%%%%%%%%%%%%%%%%%%%%%%%%%%%%%%%%%%%%%%%%%%%%%%%%%%%%%%%

\graphicspath{{./figures/}{./}}

\begin{document}

\preprint{}

\title{Phase structure of the Quark-Meson model}

\author{Rui Wen}
\affiliation{School of Physics, Dalian University of Technology, Dalian, 116024,
  P.R. China}

\author{Yong-rui Chen}
\affiliation{School of Physics, Dalian University of Technology, Dalian, 116024,
  P.R. China}

\author{Wei-jie Fu}
\email{wjfu@dlut.edu.cn}
\affiliation{School of Physics, Dalian University of Technology, Dalian, 116024,
  P.R. China}

%\date{\today}% It is always \today, today,
             %  but any date may be explicitly specified

\begin{abstract}

 


\end{abstract}

%\pacs{Valid PACS appear here}% PACS, the Physics and Astronomy
\pacs{11.30.Rd, %Chiral symmetries
          11.10.Wx, %Finite-temperature field theory
          05.10.Cc, %Renormalization group methods
          12.38.Mh  %Quark-gluon plasma
     }                             % Classification Scheme.
%\keywords{Suggested keywords}%Use showkeys class option if keyword
                              %display desired
\maketitle

%\tableofcontents

%%%%%%%%%%%%%%%%%%%%%%%%%%%%%%%%%%%%%%%%%%%%%%%%%%%%%%%%%%%%%
%%%%%%%%%%%%%%%%%%%%%%%%%%%%%%%%%%%%%%%%%%%%%%%%%%%%%%%%%%%%%
\section{Introduction}
\label{sec:intro}

A

%%%%%%%%%%%%%%%%%%%%%%%%%%%%%%%%%%%%%%%%%%%%%%%%%%%%%%%%%%%%%
%%%%%%%%%%%%%%%%%%%%%%%%%%%%%%%%%%%%%%%%%%%%%%%%%%%%%%%%%%%%%
\section{2 flavor quark meson model}
\label{sec:QM}

The effective action of the 2 flavor quark-meson model reads \cite{Fu:2015naa}
\begin{align}
	\Gamma_k [\Phi]&=\int_x\{Z_{q,k} \bar q (\gamma_\mu \partial_\mu-\gamma_0 \mu)q +\frac{1}{2} Z_{\phi,k}(\partial_\mu \phi)^2  \nonumber\\[2ex]
	&+ h_k \bar q (T^0 \sigma +i \gamma_5\vec{T} \cdot \vec{\pi} )q +V_k(\rho) - c \sigma \} \label{eq:effective action}
\end{align}
with the integral $\int_x=\int_0^{1/T}d_{x_0}\int d^3 x$. $T$ and $\mu$ denote the temperature and quark chemical potential,  respectively. The meson/quark wave function $Z_{\phi,k}/Z_{q,k}$ are scale and field dependent. Moreover, in this work we ignore the  longitudinal  and transversal  splitting of $Z_k$, i.e., $Z_k^\parallel $ and $Z_k^\perp$,  respectively, and an interesting discussion is presented in Ref. \cite{Yin:2019ebz}. $T^0$ and $\vec{T}$ are generators of $SU (N_f=2)$ with $\Tr(T^iT^j)=\frac{1}{2} \delta^{ij}$ and $T^0=\frac{1}{\sqrt{N_f}} \mathbb{1}_{N_f \times N_f}$. Quark and meson interaction is encoded in Yukawa coupling $h_k$ in \Eq{eq:effective action}. The effective potential $V_k(\rho)$ is scale and field dependent and chirally invariant with chirally invariant variable $\rho=\frac{1}{2}\phi^2$. $\phi=(\sigma ,\vec{\pi})$ is the meson field and its vacuum expectation value $\phi_0=(\sigma,0)$. The linear sigma term $- c \sigma$ explicitly breaks the chiral symmetry.

We introduce the renormalized field and renormalized coupling parameters
\begin{align}
	\bar{\phi}=Z_{\phi} ^{1/2}\phi , \quad \bar{h}=\frac{h}{Z_\phi^{1/2} Z_q}
\end{align}
And the field dependent renormalized dimensionless meson and quark masses are obtained by
\begin{align}
	\bar m_\pi&=\frac{V_k'(\rho)}{k^2 Z_{\phi,k}} \\[2ex]
	\bar m_\sigma&= \frac{V_k'(\rho)+2 \rho V_k''(\rho)} {k^2 Z_{\phi,k}}\\[2ex]
	\bar m_q&=\frac{h_k^2 \rho}{2 k^2 Z_{q,k}^2}\label{eq:mass}
\end{align}
the masses given above are curvature masses which distinguish from the pole and screening masses \cite{Herbst:2013ufa,Alkofer:2018guy}..

The QM model at various truncations have been investigated in Ref. \cite{Pawlowski:2014zaa,Fu:2015naa,Fu:2015amv,Rennecke:2016tkm,Yin:2019ebz}. However, all of these investigation are based on Taylor expansion technology which only consider field dependent at the neighborhood of vacuum expectation value $\phi_0$, while the full field dependent is not included. In \cite{Pawlowski:2014zaa}, they show a good match of chiral phase diagram between Taylor expansion solution and  grid solution within LPA truncations at low chemical potential. However, the Taylor expansion technology disables to resolve the problem at high chemical potential and low temperature.

Due to the Matsubara
 
Then with the Wetterich equation
\begin{align}
	\partial_t \Gamma_k [\Phi] = \frac{1}{2}\Tr G_{\phi \phi} [\Phi]  \partial_t R^\phi_k-\Tr G_{q \bar q}[\Phi] \partial_t R^q_k
\end{align}
with the regulators $R^\phi_k,R^q_k$ are defined in Eq{...}

\begin{align}
	\partial_t V_k(\rho) &=\frac{k^4}{4 \pi^2} \{ ( N_f^2-1)l_0^B(m_{\pi,k},\eta_{\phi,k} ; T)\nonumber\\[2ex]
	&+l_0^B(m_{\sigma,k},\eta_{\phi,k}; T) \nonumber\\[2ex]
	&-4N_c N_f l_0^F(m_{q,k},\eta_{q,k};T,\mu)\}
\end{align}
here $l_0^{B/F}$ are threshold functions which are defined in \ref{app:threshold_fun}

\begin{align}
	\eta_{\phi,k}&=\frac{1}{6\pi^2} \Big\{ \frac{4}{k^2} \rho (V_k'(\rho))^2
\end{align}
%%%%%%%%%%%%%%%%%%%%%%%%%%%%%%%%%%%%%%%%%%%%%%%%%%%%%%%%%%%%%
%%%%%%%%%%%%%%%%%%%%%%%%%%%%%%%%%%%%%%%%%%%%%%%%%%%%%%%%%%%%%
\section{Results}
\label{sec:Results}

\subsection{phase diagram at chiral limit}

%%%%%%%%%%%%%%%%%%%%%%%%%%%%%%%%%%%%%%%%%%%%%%%%%%%%%%%%%%%%%
%%%%%%%%%%%%%%%%%%%%%%%%%%%%%%%%%%%%%%%%%%%%%%%%%%%%%%%%%%%%%

\section{Summary and discussions}
\label{sec:sum}

In this work, we have studied

%%%%%%%%%%%%%%%%%%%%%%%%%%%%%%%%%%%%%%%%%%%%%%%%%%%%%%%%%%%

\begin{acknowledgments}

The work was supported by the National Natural Science Foundation of China under Contracts Nos. 11775041.

\end{acknowledgments}

%%%%%%%%%%%%%%%%%%%%%%%%%%%%%%%%%%%%%%%%%%%%%%%%%%%%%%%%%%%%%
%%%%%%%%%%%%%%%%%%%%%%%%%%%%%%%%%%%%%%%%%%%%%%%%%%%%%%%%%%%%%
\appendix

\section{Threshold Functions}
\label{app:threshold_fun}
\begin{align}
	l_0^B(\bar m_{\phi,k}^2,\eta_{\phi,k}) &=\frac{2}{3}\frac{1}{\sqrt{1+\bar m_{\phi,k}^2}} \Big ( 1-\frac{\eta_{\phi,k}}{5}\Big) \nonumber\\[2ex] 
	&\Big( \frac{1}{2}+n_B(\bar m_{\phi,k}^2,T) \Big)  \\[2ex]
	l_0^F(\bar m_{q,k}^2,\eta_{q,k})&=\frac{1}{3} \frac{1}{\sqrt{1+\bar m_{q,k}^2}} \Big(1-\frac{\eta_{q,k}}{4}\Big) \nonumber\\[2ex]
	&\Big(1-n_F(E-\mu)-n_F(E+\mu))
\end{align}


\bibliography{ref.bib}% Produces the bibliography via BibTeX.

\end{document}
%
% ****** End of file apssamp.tex ******
